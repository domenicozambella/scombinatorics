% !TEX root = scombinatorics.tex
\documentclass[scombinatorics.tex]{subfiles}
\begin{document}
\chapter{Appendix: Ramsey's Theorem}\label{ramsey}

\def\medrel#1{\parbox[t]{5ex}{$\displaystyle\hfil #1$}}
\def\ceq#1#2#3{\parbox[t]{20ex}{$\displaystyle #1$}\medrel{#2}{$\displaystyle #3$}}

\section{Upper bound for Ramsey numbers}

\begin{definition}
  The Ramsey number $R(s,t)$ is the minimum number $n$ such that any graph on $n$ vertices contains either an anticlique of size $s$ or a clique of size $t$.\QED
\end{definition}

Note that it is not clear a priori that Ramsey numbers are finite.
Indeed, it could be the case that there is no finite number satisfying the conditions of $R(s, t)$ for some choice of $s, t$. 
However, the following theorem proves that this is not the case and gives an explicit bound on $R(s, t)$.

\begin{theorem}[(Ramsey's theorem)]
  For any $s, t \ge 1$, the Ramsey number $R(s, t)$ is finite.
  In particular,

  \ceq{\#\hfill R(s,t)}{\le}{\binom{s+t-2}{s-1}}
\end{theorem}

\begin{proof}
  First we prove that $R(s, t) \le R(s-1, t) + R(s, t-1)$ holds for every $s,t>1$.
  To see this, consider a graph on $n = R(s-1, t) + R(s, t-1)$ vertices.
  We prove that this graph has an anticlicque of size $s$ or a clicque of size $t$
  Fix a vertex $v$.
  It is immediate to verify that exacly one of these two cases occur:
  \begin{itemize}
    \item[i.] There are at least $R(s, t-1)$ edges
    adjacent to $v$. 
    Then we apply the definition of $R(s, t-1)$ to the neighbors of $v$, which implies that either they contain an anticlique of size $s$, or a clique of size $t-1$.
    In the first (sub)case we are done.
    Otherwise, we can extend this clique by adding $v$, and hence the whole graph contains a clique of size $t$.
    \item[ii.] There are at least $R(s-1, t)$ that are not $v$ nor are adjacent to $v$.
    Then we apply the definition of $R(s-1, t-1, t)$ to the non-neighbors of $v$ and we get either an anticlique of size $s-1$, or a clique of size $t$.
    In the latter (sub)case we are done.
    Otherwise, the anticlique can be extended by adding $v$ hence the whole graph contains an anticlique of size $s$.
  \end{itemize}

Given that $R(s, t) \le R(s-1, t) + R(s, t-1)$, it follows by induction that all Ramsey numbers are finite. Moreover, we get an explicit bound.
First, \# holds for the base cases where $s = 1$ or $t = 1$ since every graph contains a clique or an anticlique of size $1$.
The inductive step is as follows:

\ceq{\hfill R(s,t)}{\le}{R(s-1, t) + R(s, t-1)}

\ceq{\ssf 1}{\le}{\binom{s+t-3}{s-2} + \binom{s+t-3}{s-1}}

\ceq{\ssf 2}{\le}{\binom{s+t-2}{s-1}}

where \ssf1 follows from the induction hypothesis and \ssf2 follows from a standard identity for binomial coefficients.
\end{proof}

Now we consider a more general setting.
We color the edges of $K_n$, the complete graph
on $n$ vertices, with a certain number of colors and we ask whether there is a clique of a certain size all of whose edges have the same colour.
We shall see that this is always true for a sufficiently large $n$.
Note that the question about clique and anticlique corresponds to rapresenting a graph of size $n$ as a coloring of $K_n$ with $2$ colors, adjacent and non-adjacent.

This leads to a more general definition of Ramsey numbers

\begin{definition}
  The Ramsey number $R(s_1,\dots,s_k)$ is the minimum number $n$ such that any coloring of the edges of $K_n$ with $k$ colors contains a clique of size $s_i$ in color $i$, for some $i$.\QED
\end{definition}

The following Theorem can be easily proved by induction on the number of colors.

\begin{theorem}
  For any $s_1,\dots,s_k\ge1$, the Ramsey number $R(s_1,\dots,s_k)$ is finite.\QED
\end{theorem}

%%%%%%%%%%%%%%%%%%%%%%%%%%%%%%%%%%%%%%%
%%%%%%%%%%%%%%%%%%%%%%%%%%%%%%%%%%%%%%%
%%%%%%%%%%%%%%%%%%%%%%%%%%%%%%%%%%%%%%%
%%%%%%%%%%%%%%%%%%%%%%%%%%%%%%%%%%%%%%%
%%%%%%%%%%%%%%%%%%%%%%%%%%%%%%%%%%%%%%%
\section{Lower bound for Ramsey numbers}

\def\medrel#1{\parbox[t]{5ex}{$\displaystyle\hfil #1$}}
\def\ceq#1#2#3{\parbox[t]{35ex}{$\displaystyle #1$}\medrel{#2}{$\displaystyle #3$}}

\begin{theorem}
    For every integer $k\ge 3$
    
    \ceq{\hfill\lfloor2^{k/2}\rfloor}{<}{R(k,k).}
\end{theorem}

\begin{proof}
  Fix some $k\ge3$. Consider a graph on $n$ vertices obtained by randomly choosing whether or not to put an edge between two verteces.
  Each choice is taken independently by tossing a fair coin.

  For every set $K$ of $k$ verteces, the probability that $K$ is a clique is 

  \ceq{\hfill\Pr(K\textrm{ is a clique})}{=}{2^{-\binom{k}{2}}.}

  By symmetry, this equals also $\Pr(K\textrm{ is an anticlique})$, hence

  \ceq{\hfill\Pr(K\textrm{ is a clique or an anticlique})}{=}{2^{1-\binom{k}{2}}}

  The probability that there is a clique or an anticlique of size $k$, is 

  \ceq{}{\le}{\binom{n}{k}\ 2^{1-{\binom{k}{2}}}}

  Hence, if for some $n$ the function above is $<1$, some graph of size $n$ with no clique nor anticlique exists. 
  It is only let to verity that this occurs for all $n\le\lfloor2^{k/2}\rfloor$ 

  \ceq{\hfill\binom{n}{k}\ 2^{1-{\binom{k}{2}}}}{=}{\frac{n!}{k!(n-k)!}\ 2^{1-\frac{k(k-1)}{2}}}

  \ceq{}{=}{\binom{n}{k}\ \frac{2^{1+k/2}}{2^{k^2/2}}}
  
  \ceq{}{<}{\frac{n^k}{k!}\ \frac{2^{1+k/2}}{2^{k^2/2}}}
  
  \ceq{}{<}{1}

The last inequality holds because $2^{1+k/2}<k!$ for $k\ge3$ and, when $n$ is as above, $n^k\le\lfloor2^{k/2}\rfloor^k\le 2^{k^2/2}$.
\end{proof}

See \cite{AS} for a discussion of the method of proof.

\end{document}